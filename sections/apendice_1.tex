	\section{Planejamento da Revisão Sistemática da Literatura}
\begin{table}[h!]
	\centering
    \caption{Objetivo do estudo.}
    \label{tab:rsl}
	\begin{tabularx}{\textwidth}{|l|X|}
    	\hline
		\textbf{Analisar} & publicações científicas de um estudo baseado em revisão sistemática \\
        \hline
        \textbf{Com o propósito de} & identifica-las \\
        \hline
        \textbf{Em relação às} & vantagens e desvantagens da utilização de uma luva inteligente para a captura de movimentos para controle de dispositivos eletro-eletrônicos \\
        \hline
        \textbf{Do ponto de vista do} & pesquisador \\
        \hline
        \textbf{No contexto} & acadêmico, industrial, residencial ou urbano para a interação com dispositivos inteligentes ou não. \\
        \hline
	\end{tabularx}

\end{table}

\textbf{Formulação das perguntas.} Buscamos responder as seguintes perguntas:

\begin{itemize}
	\item Q1: Quais são os principais métodos de captura de movimentos utilizando luvas inteligentes?
    \begin{itemize}
    	\item Q1.1:  Qual o sensor ou combinação de sensores que se mostra mais preciso na captura dos movimentos de mão e punho?
        \item Q1.2: Qual a melhor disposição de sensores para a confecção da luva?
        \item Q1.3: Quais as limitações dos métodos propostos?
    \end{itemize}
    \item Q2: Qual o algoritmo ou combinação de algoritmos mais eficiente para classificação dos movimentos?
    \begin{itemize}
    	\item Q2.1 Qual o volume suficiente de dados para uma classificação ótima dos movimentos?
    \end{itemize}
    \item Q3: Qual o melhor protocolo de comunicação entre vestíveis e dispositivos inteligentes em uso atualmente?
\end{itemize}

\textbf{Escopo da Pesquisa.} Para delinear o escopo da pesquisa foram estabelecidos critérios para garantir, de forma equilibrada, a viabilidade da execução (custo, esforço e tempo), acessibilidade aos dados e abrangência do estudo. A pesquisa acontecerá a partir de bibliotecas digitais através das suas respectivas máquinas de busca e, quando os dados não estiverem disponíveis eletronicamente, através de consultas manuais. 

\textbf{Critérios de Seleção de Fontes.} Para as bibliotecas digitais é necessário:

\begin{itemize}
	\item Possuir máquina de busca que permita o uso de expressões lógicas ou mecanismo equivalente;
    \item Incluir em sua base publicações da área de exatas ou correlatas que possuam relação direta com o tema
a ser pesquisado;
	\item As máquinas de busca deverão permitir a busca no texto completo das publicações.
\end{itemize}

Além disso, os mecanismos de busca utilizados devem garantir resultados únicos através da busca de um mesmo conjunto de palavras-chaves (string de busca). Quando isto não for possível, deve-se estudar e documentar uma forma de minimizar os potenciais efeitos colaterais desta limitação.

\todo{Arrumar as referências das editoras.}
\textbf{Métodos de Busca das Publicações.} As fontes digitais foram acessadas via Web, através de expressões de busca pré-estabelecidas. A biblioteca digital consultada foi a Scopus, acessível em http://www.scopus.com. Segundo a editora Elsevier (2013), a Scopus é uma das maiores bases de dados de resumos e citações da literatura de pesquisa peer-reviewed com mais de 20,500 títulos de mais de 5,000 editoras internacionais. Dentre estas editoras podemos citar: Springer (Springer (2013)); IEEE Xplore Digital Library (IEEE (2013)); ACM Digi- tal Library (ACM (2013)); ScienceDirect/Elsevier (B.V (2013)); Wiley Online Library (Sons (2013)); British Computer Society (Society (2013)) dentre outras. A biblioteca Scopus também inclui aproximadamente 5.3 milhões de conferências de artigos de proceedings e journals, 400 publicações comerciais, 360 série de livros e publicações aceitas é disponibilizado online antes da publicação oficial em mais de 3,850 periódicos. Ainda segundo a editora Elsevier (2013), a Scopus tem aproximadamente 2 milhões de novas gravações adicionadas
a cada ano com atualizações diárias.

\todo{Arrumar as referências da string de busca.}
\textbf{String de Busca.} A string de busca foi definida segundo o padrão PICO (do inglês Population, Intervention,
Comparison, Outcomes) (Kitchenham and Charters (2007)), conforme a estrutura abaixo:

\begin{itemize}
	\item População: Trabalhos publicados em conferências e periódicos que são relacionados com dispositivos vestíveis;
    \item Intervenção: Relações entre dispositivos vestíveis baseados em capturas de movimentos de mão;
    \item Comparação: Análise das diferentes abordagens para captura e classificação de movimentos, no sentido de medir a abrangência de cada abordagem diante das métricas propostas não levando em consideração sua eficácia e desempenho, utilizando as questões de pesquisa como fonte para a extração de métricas;
    \item Resultados: A partir dos relatos das diferentes abordagens de captura de movimentos de mão, pretende-se verificar a eficácia de cada método no contexto de controle de dispositivos eletro-eletrônicos.
\end{itemize}

Como este estudo representa um estudo de mapeamento/caracterização, a string de busca (para execução na biblioteca digital Scopus como mencionado anteriormente) foi definida de acordo com dois aspectos: População e Intervenção (Peterson et al., 2008), como é apresentado na estrutura abaixo.

\begin{itemize}
	\item População: publicações que fazem referências à dispositivos baseados em luvas (e sinônimos):
    \begin{itemize}
    	\item "wearable sensors" OR "wearable interaction devices" OR "smart watch" OR "glove-based systems" OR "glove systems" OR "translation glove" OR "unobtrusive wearable" OR "interaction devices" OR "man-machine interfaces"
    \end{itemize}
    \item Intervenção: captura e análise de movimentos (e sinônimos):
    \begin{itemize}
    	\item "analysis of gestures" OR "hand movement data" OR "gesture recognition" OR "hand gesture interface" OR "hand gesture recognition" OR "gesture recognition framework" OR "gesture-based interaction" OR "personalized gesture" OR "gesture interactions" OR "gesture set"   
    \end{itemize}
\end{itemize}

\subsection{Procedimentos de Seleção e Critérios}

A estratégia de busca será aplicada por um pesquisador para identificar as publicações em potencial. A seleção das publicações ocorrerá em 3 etapas:

\begin{enumerate}
	\item \textbf{Seleção e catalogação preliminar dos dados coletados.} 
    A seleção preliminar das publicações será feita
a partir da aplicação da expressão de busca às fontes selecionadas. Cada publicação será catalogada em um banco de dados criado especificamente para este fim e armazenada em um repositório para análise posterior;

    \item \textbf{Seleção de dados relevantes - [1º filtro].}
    A seleção preliminar com o uso da expressão de busca não
garante que todo o material coletado seja útil no contexto da pesquisa, pois a aplicação das expressões de
busca é restrita ao aspecto sintático. Dessa forma, após a identificação das publicações através dos mecanismos de buscas, deve-se ler o título, os resumos/abstracts e as palavras-chave e analisá-los seguindo os critérios de inclusão e exclusão identificados a seguir. Neste momento, poder-se-ia classificar as publicações apenas quanto aos critérios de exclusão, entretanto, para facilitar a análise e reduzir o número de publicações das quais se possam ter dúvidas sobre sua aceitação, deve-se também classificá-las quanto aos critérios de inclusão. Devem ser excluídas as publicações contidas no conjunto preliminar que:
	\begin{itemize}
		\item \textbf{CE1-01:} Não serão selecionadas publicações em que as palavras-chave da busca não apareçam no título, resumo e/ou texto da publicação (excluem-se os seguintes campos: as seções de agradecimentos, biografia dos autores, referências bibliográficas e anexas).
        \item \textbf{CE1-02:} Não serão selecionadas publicações em que descrevam e/ou apresentam ‘keynote speeches’, tutoriais, cursos e similares.
        \item \textbf{CE1-03:} Não serão selecionadas publicações em que não se utilize um dispositivo vestível para a captura de movimentos de mão.
	\end{itemize}
	Podem ser incluídas apenas as publicações contidas no conjunto preliminar que:
    \begin{itemize}
    	\item \textbf{CI1-01:} Podem ser selecionadas publicações em que no contexto das palavras-chave utilizadas no artigo levem a crer que a publicação cita um método de captura de movimentos de mão utilizando um dispositivo vestível.
        \item \textbf{CI1-02:} Podem ser selecionadas publicações em que no contexto das palavras-chave utilizadas no artigo levem a crer que a publicação cita diferentes modelos de captura utilizando dispositivos vestíveis recomendando o mais preciso para a aplicação em questão.
    \end{itemize}
    \item \textbf{Seleção de dados relevantes - [2º filtro].}
    Apesar de limitar o universo de busca, o 1º filtro empregado não garante que todo o material coletado seja útil no contexto da pesquisa. Por isso, após a leitura na íntegra dos artigos selecionados no 1º filtro, deve-se verificar se as publicações respeitam os critérios abaixo. O objetivo deste 2º filtro é identificar artigos que relacionam a utilização de um dispositivo vestível para a captura de movimentos
    \begin{itemize}
    	\item \textbf{CS2 -VES -CAP\_MOV} Não devem ser selecionadas publicações que contextualizam dispositivos vestíveis com captura de movimentos.
        \item \textbf{CS2 +VES -CAP\_MOV} Não devem ser selecionadas publicações que citam dispositivos vestíveis, mas não realizam captura de movimentos.
        \item \textbf{CS2 -VES +CAP\_MOV} Não devem ser selecionadas publicações que não utilizem dispositivos vestíveis, mas utilizem captura de movimentos.
    \end{itemize}
    Dessa forma, todas as publicações devem respeitar o critério abaixo:
    \begin{itemize}
    	\item \textbf{CI2 +VES +CAP\_MOV} Só podem ser selecionadas publicações que utilizem dispositivos vestíveis para realizar captura de movimentos.
    \end{itemize}
\end{enumerate}



