\label{chapter:consi}

O método desenvolvido para a Hand.io atingir os objetivos definidos na \autoref{chapter:intro} deste trabalho, se mostra eficaz simples. A maneira que será realizado o reconhecimento de gestos e a execução de ações parecem estar bem definida e a escolha das ferramentas de modelagem e verificação automática do protótipo já estão em estudos avançados. 

O levantamento da base de dados necessária para realizar o treinamento dos algoritmos de aprendizado de máquina utilizados na Hand.io, e a definição de um vocabulário de gestos simples e coeso irão necessitar de grandes esforços por parte do autor e de voluntários dispostos a contribuir com este trabalho. 

O desenvolvimento do protótipo se encontra bem encaminhado, com definição e a aquisição dos componentes necessários já realizada. O que permitirá ao autor deste trabalho realizar testes extensivos com os sensores e atuadores que irão compor o protótipo do sistema.

Esperasse que ao final do cronograma definido na \autoref{tab:cronograma} todos os objetivos deste trabalho tenham sido cumpridos a tempo da realização da apresentação final deste trabalho de conclusão de curso.
