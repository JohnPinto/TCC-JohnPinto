\label{chapter:consi}

Este trabalho apresentou uma proposta para projetar e avaliar um sistema computacional para uma luva inteligente efetuar a comunicação com dispositivos eletrônicos por meio do reconhecimentos de padrões (por meio da aplicação de técnicas de aprendizagem de máquina) de gestos manuais. 
% , utilizando métodos e técnicas no processo de codificação e prototipação deste dado sistema, visando garantir os requisitos de previsibilidade; confiabilidade; e baixo custo. 
Assim, focando na criação de uma interface única e intuitiva entre um usuário e um ambiente inteligente, visando facilitar o uso de dispositivos através de gestos, que são um meio natural de comunicação.
% O método desenvolvido para a Hand.io atingir os objetivos definidos na \autoref{chapter:intro} deste trabalho, se mostra eficaz simples. A maneira que será realizado o reconhecimento de gestos e a execução de ações parecem estar bem definida e a escolha das ferramentas de modelagem e verificação automática do protótipo já estão em estudos avançados. 

O levantamento da base de dados necessária para realizar o treinamento dos algoritmos de aprendizado de máquina utilizados na Hand.io, e a definição de um vocabulário de gestos simples e coeso são os próximos passo deste trabalho, visto que irão necessitar de grandes esforços por parte do autor e de voluntários dispostos a contribuir com este trabalho. 

O desenvolvimento do protótipo se encontra em progresso, com definição e a aquisição dos componentes necessários já realizada. O que permitirá ao autor deste trabalho realizar testes extensivos com os sensores e atuadores que irão compor o protótipo do sistema.

% Esperasse que ao final do cronograma definido na \autoref{tab:cronograma} todos os objetivos deste trabalho tenham sido cumpridos a tempo da realização da apresentação final deste trabalho de conclusão de curso.
