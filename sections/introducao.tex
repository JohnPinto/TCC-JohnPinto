\label{chapter:intro}
A complexidade de sistemas computacionais cresce exponencialmente, visando aplicações em diferentes domínios, tais como: agricultura, automação residencial e robótica. Neste sentido, Sistemas Embarcados (SE) que são sistemas computacionais integrados a outros sistemas, usualmente são encontrados no nosso dia a dia. Geralmente, o principal propósito dos SE é o controle e provimento de informações para uma função específica \cite{RAMESH:2012} sedo estes extremamente iterativos com seu ambiente, operam geralmente em tempo real, e estão disponíveis continuamente. 


Segundo \citeonline{SERGEY:2016}, robôs se tornam mais inteligentes com respostas em tempo real. Sistemas como o Controle Integrado de Forças processam as variações com sensitividade humana, melhorando a performance, e reduzindo assim o tempo de codificação. Isso torna os robôs mais inteligentes e capazes, como os humanos, de manusear itens com interações externas em tempo real. Robôs podem auxiliar pessoas com restrições motoras a realizar atividades cotidianas que muitas vezes envolvem o contato com o rosto de uma pessoa.


Levando em consideração estes avanços no campo de sensores robóticos, desenvolver sistemas embarcados com características robóticas, por exemplo, a capacidade de interação com o ambiente, pode ser uma solução viável para que humanos possam interagir com dispositivos eletro-eletrônicos (splits de ar-condicionado, televisões, computadores e outros) ao seu redor sem a necessidade do toque presencial. Um modo de tornar esta solução viável seria a utilização de luvas inteligentes, as quais contam com utilização de: acelerômetros, sensores de força, um microcontrolador e transmissores \cite{WESTERFELD:2012} para reconhecimento de padrões e então comunicação com os dispositivos eletro-eletrônicos \cite{OFLYNN:2013} \cite{BERNIERI:2015} \cite{CHOUDHARY:2015}. Adicionalmente, tais características robóticas podem auxiliar na recuperação de doenças ou mesmo prover suporte complementar em casos de necessidades especiais dos humanos, exemplo, deficiência visual.


No trabalho de \citeonline{CHOUDHARY:2015} é proposta uma nova abordagem para auxiliar a comunicação e a interação de indivíduos surdos-mudos, aumentando sua independência. Este inclui uma luva inteligente que traduz o alfabeto Braille, que é o mais universalmente utilizado pela população surdo-muda alfabetizada, em texto e vice-versa, e comunica mensagens via SMS para contato remoto. Isto permite ao usuário transmitir mensagens simples através de sensores de toque capacitivos como sensores de entrada colocados na palma da luva e convertidos para texto pelo computador/telefone móvel. O usuário pode entender e interpretar mensagens recebidas através de padrões de retornos táteis de pequenos motores de vibração no dorso da luva. A implementação bem-sucedida da tradução bidirecional em tempo real entre Inglês e Braille, e a comunicação entre o dispositivo vestível e o computador/telefone móvel abrem novas oportunidades para a troca de informação que até então estavam indisponíveis para indivíduos surdos-mudos, como a comunicação remota, assim como transmissões paralelas de um pra muitos.


\citeonline{OFLYNN:2013} apresenta o desenvolvimento de uma luva inteligente para facilitar o processo de reabilitação de Artrite Reumatoide \cite{ar:2008} através da integração de sensores, processadores e tecnologia sem fio para medir empiricamente o alcance do movimento. A Artrite Reumatoide é uma doença que ataca o tecido sinovial lubrificando as juntas do esqueleto. Esta condição sistêmica afeta os sistemas esquelético e muscular, incluindo ossos, juntas, músculos e tendões o que contribui para a perda de função de alcance do movimento. Medições tradicionais de artrite precisam de exames pessoais intensamente trabalhosos realizados por uma equipe médica que através de suas medições objetivas podem dificultar a determinação e a análise da reabilitação da artrite. A luva proposta \cite{OFLYNN:2013} usa uma combinação de 20 sensores de dobras, 16 acelerômetros de três eixos, e 11 sensores de força para detectar o movimento de juntas. Todos os sensores são posicionados em uma placa de circuito impresso flexível para permitir um alto nível de flexibilidade e estabilidade do sensor.


Analisando o processo de desenvolvimento destes sistemas, verifica-se que existem diferentes graus de complexidades tanto no software quanto no hardware. Desta forma, é necessário que as aplicações sejam projetadas considerando os requisitos de previsibilidade e confiabilidade, principalmente em aplicações de sistemas embarcados críticos, onde diversas restrições (por exemplo, tempo de resposta e precisão dos dados) devem ser atendidas e mensuradas de acordo com os requisitos do usuário, caso contrário uma falha pode conduzir a situações catastróficas. Por exemplo, o erro de cálculo da dose de radiação no Instituto Nacional de Oncologia do Panamá que resultou na morte de 23 pacientes \cite{WONG:2010}. Contudo, erros durante o desenvolvimento de sistemas computacionais tornam-se mais comuns, principalmente quando se tem curto espaço de tempo de liberação do produto ao mercado, e estes sistemas precisam ser desenvolvidos rapidamente e atingir um alto nível de qualidade. 


Visando contribuir com metodologias de desenvolvimento de sistemas embarcados, o contexto deste trabalho está situado em demonstrar métodos e técnicas no processo de codificação e prototipação de sistemas embarcados, tais como: Redes de Petri \cite{BENDERS:1992} \cite{VALK:2002}, UML \cite{ROCHA:2011}; e Máquinas de Estados \cite{LAMPKA:2009}. A utilização destas ferramentas garante à um sistema uma confiabilidade muito maior, por serem métodos padronizados com um certo formalismo matemático que podem ser testados de forma automatizada através de simulações. Sistemas que são utilizados praticamente durante o dia todo como peças de roupas precisam ter seu funcionamento garantido para que tenham utilidade ao usuário.


Sistemas integrados à roupa do usuário passarão a ser comuns na interação entre um usuário e seu ambiente, seja ele residencial ou urbano. Segundo \citeonline{Weiser:1999} as redes ubíquas já são capazes de realizar certas operações em um contexto onde diversos dispositivos inteligentes estão em constante comunicação. Em um futuro próximo objetos cotidianos e aparelhos eletro-eletrônicos passarão a ser capazes de se comunicarem uns com os outros, e compartilhar dados de seus sensores com outros sistemas que podem ser utilizados para mudar completamente a relação dos usuários com o seu ambiente. Existem aplicações nos campos de médicos, automação industrial, gerenciamento inteligente de energia, assistência à idosos, e muitos outros. \cite{iotct:2014}


Neste contexto, este trabalho visa o desenvolvimento de uma luva inteligente de baixo custo para controle de dispositivos eletrônicos por meio do reconhecimento de padrões de movimentos, visto que tais aplicações se mostra de grande utilidade para o usuário em um ambiente inteligente, assim provendo conforto e facilidades ao seu usuário. Adicionalmente, para o desenvolvimento do sistema proposto, este trabalho visa projetar e analisar componentes de hardware e software com o foco em IoT que tenham um baixo consumo de energia; baixo custo de implementação; e aplicação de métodos para garantir a qualidade da execução do sistema computacional proposto.
% e métodos para otimizar a codificação e verificação de sistemas embarcados.


\section{Definição do Problema}

O problema considerado neste trabalho é expresso na seguinte questão: Como projetar e desenvolver um sistema embarcado, para uma luva inteligente de baixo custo, utilizando métodos e técnicas para garantir os requisitos de previsibilidade e confiabilidade do sistema, de tal forma que o sistema proposto auxilie o seu usuário na comunicação com dispositivos eletrônicos?


% Quais métodos e técnicas podem ser utilizados para o desenvolvimento de um sistema embarcado, em uma luva inteligente, visando garantir os requisitos de previsibilidade e confiabilidade do sistema?


\section{Objetivos}

% O objetivo principal deste trabalho é demonstrar métodos e técnicas para projetar e desenvolver um sistema embarcado em uma luva inteligente que seja capaz de reconhecer padrões de movimentos/gestos para efetuar o controle de dispositivos eletro-eletrônicos, utilizando componentes de baixo custo. 

O objetivo principal deste trabalho é projetar e avaliar um sistema computacional para uma luva inteligente efetuar a comunicação com dispositivos eletrônicos por meio do reconhecimentos de padrões de gestos manuais, utilizando métodos e técnicas no processo de codificação e prototipação deste dado sistema, visando garantir os requisitos de previsibilidade; confiabilidade; e baixo custo. Assim, visando a criação de uma interface única e intuitiva entre um usuário e um ambiente inteligente, visando facilitar o uso de dispositivos através de gestos, que são um meio natural de comunicação.

% O objetivo principal deste trabalho é demonstrar métodos e técnicas no processo de codificação e prototipação de sistemas embarcados em uma luva inteligente, visando garantir os requisitos de previsibilidade; confiabilidade; e baixo custo para a comunicação com dispositivos eletrônicos por meio do reconhecimentos de padrões de gestos. Assim, visando a criação de uma interface única e intuitiva entre um usuário e um ambiente inteligente, visando facilitar o uso de dispositivos através de gestos, que são um meio natural de comunicação.



\subsection{Objetivos Específicos}
\begin{enumerate}
    \item Identificar métodos para a modelagem do software e hardware;
    \item Definir um modelo formal para o fluxo do funcionamento da luva, visando analisar propriedades de segurança do fluxo de execução do sistema;
    \item Demonstrar uma técnica para transformação de modelos de software em códigos para o projeto;
    \item Projetar e desenvolver uma central de controle que será o meio de comunicação entre os dispositivos eletrotônicos disponíveis em um ambiente com e a luva para obtenção dos dados de movimentos;
    \item Identificar e aplicar um protocolo de comunicação entre os dispositivos eletrotônicos disponíveis em um ambiente com a luva;
    \item Realizar o levantamento de componentes eletrônicos com baixo consumo de energia e baixo custo necessários para o desenvolvimento de um protótipo;    
    \item Propor um algoritmo para reconhecimento e classificação dos movimentos/sinais enviados pela luva na mão do usuário;
    \item Desenvolver um protótipo da luva e da central de controle; e
    \item Validar o método proposto pela análise da prototipação do sistema proposto, a fim de examinar a sua eficácia e aplicabilidade. 
\end{enumerate}

\section{Organização do trabalho}

Este trabalho é organizado em capítulos que servem de base para a resposta dos problemas de pesquisa deste trabalho.

No \textbf{\autoref{chapter:intro}: Introdução} foi apresentada uma contextualização do trabalho, a definição do problema e os objetivos deste trabalho.

No \textbf{\autoref{chapter:conceitos}: Conceitos e definições} são apresentadas as ferramentas que serão utilizadas no método.

No \textbf{\autoref{chapter:correlatos}: Trabalhos correlatos} são apresentados trabalhos similares a este e como estes trabalhos contribuíram para a definição do método.

No \textbf{\autoref{chapter:metodo}: Método proposto} é definido como os objetivos deste trabalho serão atingidos.

No \textbf{\autoref{chapter:cronograma}: Cronograma} está definido em quanto tempo cada fase do método irá levar para ser implementada.

\todo[inline]{Falta adicionar as considerações parciais}
% \section{Motivação}

% Este trabalho de conclusão de curso é motivado pela ideia da criação de uma interface única e intuitiva entre um usuário e um ambiente inteligente, visando facilitar o uso de dispositivos através de gestos, que são um meio natural de comunicação.

% \todo{Terminar essa seção}

%\section{Metodologia Proposta}

% \section{Contribuições Propostas}
% 
% Este trabalho busca realizar contribuições nos campos de ambientes inteligentes, internet das coisas e dispositivos vestíveis, ao propor uma interface homem-máquina simplificada, intuitiva e de baixo custo. 
