Neste capítulo serão expostos trabalhos que contam com métodos de captura de movimentos de mão utilizando dispositivos com funções similares ao trabalho aqui proposto para a construção da luva Hand.io. Para uma melhor comparação, os trabalhos selecionados contam com sensores de acelerômetros e giroscópios como principal método de captura de sinais, assim como a luva Hand.io. 
% Os métodos dos estudos serão analisados de maneira critica afim de permitir uma comparação entre os pontos fortes e fracos de cada abordagem em relação a este trabalho.

% Os trabalhos aqui analisado contam com diferentes abordagens para o registro dos gestos, sendo eles: \textit{Accelerometer-based gesture control for a design environment}~\cite{accelerometer:2006}, que propõe um pequeno dispositivo para controle de um estúdio de design inteligente; \textit{I'm home}~\cite{imhome:2011} que apresenta um método bem detalhado para a definição um dicionário de gestos eficientes utilizando a participação dos usuários durante o período de desenvolvimento do sistema; \textit{uWave}~\cite{uwave:2009} que demonstra um algoritmo eficiente de reconhecimento de gestos com uma baixa necessidade de recursos computacionais; e por fim \textit{A survey of glove-based systems and their applications}~\cite{survey:2008} faz uma análise profunda no campo de dispositivos vestíveis, em específico luvas de controle, mostrando desde os dispositivos mais primitivos até os utilizados atualmente. 

Os trabalhos aqui analisados demonstram a viabilidade do desenvolvimento da Hand.io e sua aplicação em um ambiente real. As principais particularidades da Hand.io seguem listadas abaixo e servem de parâmetro de comparação com as demais aplicações:
\begin{itemize}
    \item Captura passiva de gestos;
    \item Sensores presos à mão;
    \item Aplicado em um ambiente residencial;
    \item Presença de sensor acelerômetro;
    \item Presença de sensor giroscópio;
    \item Conexão sem fio;
    \item Controle por infravermelho; e
    \item Personalização de gestos.
\end{itemize}


\section{Accelerometer-based gesture control for a design environment}

% O artigo escrito por
O trabalho de \citeonline{accelerometer:2006} conta com dois estudos principais, sendo o primeiro a aplicação e o estudo da viabilidade da utilização de gestos em um estúdio de projetos inteligentes, em específico em aplicações de Desenho Assistido por Computador (CAD) e o segundo uma avaliação e comparação da interface por gestos em relação a outras modalidades como: controle por voz, objetos físicos utilizando RFID, uma tela de toque através de um \textit{tablet}, e um dispositivo conhecido como IntelliPen, que funciona com um lazer apontado para uma tela que realiza funções similares à um mouse. Foi utilizado um Modelo Oculto de Markov (HMM)~\cite{hmm:1989} a partir de sinais discretos para o reconhecimento dos padrões, o algoritmo conta com duas fases distintas de treino e de reconhecimento.


O dispositivo utilizado para captura de movimentos foi uma \textit{SoapBox}~\cite{soapbox:2002}, que é uma placa miniaturizada do tamanho de uma caixa de fósforos que conta com um processador, um giroscópio, um acelerômetro, um compasso eletrônico, e comunicação com e sem fios. O protótipo em questão conta com dois botões que são pressionados no inicio e no fim da realização de um gesto, além do modo de reconhecimento de gestos o protótipo conta com um modo de leitura contínua de movimentos, que pode ser acionado pressionando ambos os botões. Este modo realiza medições diretas dos valores fornecidos pelos sensores e os utiliza para ações de zoom e rotação de objetos virtuais 3D, permitindo que o usuário sinta que está segurando o objeto em suas mãos.

Os resultados do estudo apontam que existe um grande potencial para a utilização de controle por gestos em um ambiente de trabalho, em específico para o aumento da produtividade em determinadas atividades. De acordo com o grau de conhecimento dos participantes dos testes houveram diferentes preferências quanto a interface de utilização do estúdio inteligente, participantes com um grau maior de experiência com design se mostraram mais favoráveis a IntelliPen, devido a similaridade à um mouse, já os demais também demonstraram um interesse maior pela interface por gestos. 
% 
O estudo conclui que a utilização de gestos combinados com outros métodos de controle pode ser benéfica para o aumento da produtividade em um ambiente de design, por permitir um controle mais natural e intuitivo de objetos 3D e facilitar a realização de atalhos em programas CAD.

Existem diversos pontos de divergência entre o trabalho apresentado e a Hand.io, sendo o mais perdominante o fato dos sensores da Hand.io estarem presos diretamente à mão do usuário. A natureza vestível da luva Hand.io permite ao usuário controlar os dispositivos ao seu redor de maneira mais natural, sem que seja necessário um dispositivo a parte. \citeonline{accelerometer:2006} demonstra que existe espaço para controle por gestos em diversos ambientes eu que sistemas como a Hand.io podem aumentar cosideravelmente a produtividade em ambientes de trabalho.

\begin{comment}
\begin{table}[ht]
	\centering
    \caption{Comparação entre Hand.io e SoapBox}
    \label{tab:comp_soapbox}
	\begin{tabular}{|l|c|c|}
    	\hline
        %\rowcolor{\color{gray}}
		\textbf{Recursos} & \textbf{Hand.io} & \textbf{SoapBox} \\
        \hline
        \hline
        Preso à mão & x &  \\
        \hline
        Utiliza botões &  & x \\
        \hline
        Captura contínua de sinais & x &  \\
        \hline
        \hline
        \textbf{Sensores} & & \\
        \hline
        \hline
        Acelerômetro & x & x \\
        \hline
        Giroscópio & x & \\
        \hline
        \hline
        \textbf{Contexto aplicado} & Residencial & Estúdio de design \\
        \hline
	\end{tabular}

\end{table}
\end{comment}
% \todo{fazer tabela de comparação}



\section{I'm home: Defining and evaluating a gesture set for smart-home control}

\label{cor:home}

% O artigo \textit{I'm home} de 
\citeonline{imhome:2011} apresenta uma metodologia extensa para definição e avaliação de gestos para controle de ambientes inteligentes, em específico casas inteligentes. O dispositivo escolhido para a realização da captura dos gestos foi um \textit{Smartphone Iphone}, tal dispositivo foi escolhido devido a presença de um sensor acelerômetro e giroscópio e a sua grande disponibilidade. Os sinais enviados pelos sensores foram processador por um algoritmo Fast Dynamic Time Warping - FastDTW~\cite{Salvador:2007} que conta com uma complexidade de tempo de espaço linear.
% 
Foram realizados três estudos que buscavam definir um vocabulário de gestos simples e coeso, definir o grau de distinção entre diferentes gestos e o grau de dificuldade na memorização de cada gesto. 


No primeiro estudo os participantes foram colocados em um quarto com diversos dispositivos de maneira que todos os dispositivos controláveis estivessem no angulo de visão dos voluntários em todos os momentos. As ações realizadas pelo sistema foram realizadas pelos experimentadores através de uma interface gráfica seguidos por uma descrição do comando executado. Os participantes então, foram perguntados qual gesto seria mais apropriado paro comando realizado, os movimentos foram gravados com duas câmeras e foram guardados para avaliação do tempo e articulação de cada gesto.
% 
Após a realização de cada gesto foi realizada uma pesquisa para saber a compatibilidade do gesto com a ação realizada e para saber a facilidade da execução do gesto em questão. Foram realizadas $23$ ações diferentes que controlavam dispositivos diferentes.


Os resultados deste estudo apontam que a utilização de gestos pré definidos e gestos personalizados são a melhor abordagem no controle por gestos. A utilização de uma grande base de usuários durante o desenvolvimento facilita grandemente a concepção de gestos que possam ser simples e coesos para as atividades propostas. Experiências prévias com interfaces de controle, como a interruptores e reguladores rotativos, podem influenciar bastante o tipo de gesto escolhido pelos usuários. Foi constatado também que nem o tamanho do gesto nem o tempo necessário para a realização são boas referências para a distinção entre gestos realizados pelos usuários. 


No segundo estudo foram exibidas gravações dos gestos realizados no estudo anterior e logo depois foram perguntadas quais as melhores ações poderiam ser realizadas a partir daquele gesto. O ideal seria que os gestos apresentados remetessem aos comandos para os quais eles foram criados. Os gestos que fossem corretamente relacionados às ações mais vezes seriam os mais adequados para a atividade em questão. 

No estudo apresentado foi constatado que gestos mais simples tendem a ser relacionados com mais facilidade com a ação para qual estes foram criados. Movimentos intuitivos como para cima ou para baixo foram relacionados com mais facilidade à atividade que tinham sido propostos, gestos simbólicos como sinais de interrogação também tiveram bons resultados.

No terceiro estudo participantes que não tiveram relação nenhuma com os estudos anteriores. No primeiro momento os voluntários foram convidados a assistir um vídeo para cada gesto e sua respectiva ação. Cada gesto foi repetido $5$ vezes por cada participante utilizando um \textit{smartphone} que registrou os sinais dos gestos que vão ser utilizados como referência para um classificador. Os participantes então, foram perguntados qual o grau de compatibilidade entre o gesto e a ação relacionada a ele.

No segundo momento foram exibidos aos participantes somente as ações que cada gesto realizava em ordem aleatória e durante os $5$ segundos seguintes foi requisitado que o gesto correspondente à ação fosse realizado. Caso o gesto fosse correto a apresentação continuaria, caso contrário, o gesto correto seria mostrado e a ação seria exibida ao final da apresentação, que continuaria até que todos os gestos fossem realizados corretamente. Foram registradas a quantidade de erros para cada gesto.

O terceiro estudo conclui que os usuários conseguem distinguir gestos intuitivos de gestos menos intuitivos. Algo que foi comprovado pela quantidade de erros durante o teste de memorabilidade. Gestos mais simples e que rápidos são os mais indicados.

A metodologia de definição de gestos apresentada por \citeonline{imhome:2011} e o estudo da viabilidade de cada um, servem de guia para a definição dos gestos que serão utilizados na Hand.io. Testes com grupos de voluntários com diversos graus de conhecimento serão realizados durante o desenvolvimento dos gestos e em um momento posterior a viabilidade dos gestos realizados será medida.

%\todo[inline]{Descrever o que do trabalho apresentado pode ser utilizado no seu TCC}


\section{uWave: Accelerometer-based personalized gesture recognition and its applications}
\label{cor:uwave}

O trabalho realizado por \citeonline{uwave:2009} propõe um algoritmo de reconhecimento de gestos chamado \textit{uWave}, que é baseado no algoritmo \textit{Dynamic Time Warping} - DTW \cite{Salvador:2007}, diferentemente de métodos estatísticos como o Modelo Oculto de Markov - HMM~\cite{hmm:1989} que requer um volume muito grande de amostras de referência, o \textit{uWave} necessita de apenas uma amostra para realizar o reconhecimento de um gesto, algo que é muito vantajoso para a criação de gestos personalizados especificamente para cada usuário. O algoritmo foi criado para ser aplicado em qualquer dispositivo que conte com um sensor acelerômetro de três eixos, tal sensor foi escolhido devido à sua grande disponibilidade em dispositivos eletrônicos de consumo, o que aumenta consideravelmente a aplicabilidade do algoritmo em um ambiente real.

% No artigo são realizadas 
\citeonline{uwave:2009} apresenta duas aplicações distintas do algoritmo afim de demonstrar a sua versatilidade e usabilidade em cenários reais, a primeira tem como objetivo a identificação simples e precisa de um usuário através de gestos personalizados, analogamente à uma senha alfanumérica. Nesta aplicação foi utilizado um controle de um Nintendo Wii \cite{wii} como método de entrada de gestos. O estudo constata, através de uma pesquisa realizada com os usuários, que um gesto personalizado é um meio de identificação muito mais fácil de memorizar do que uma senha composta por vários caracteres. 

A segunda aplicação é a utilização de gestos para navegação de um ambiente 3D, foi desenvolvida uma rede social baseada em compartilhamento de vídeos utilizando um \textit{smartphone} da marca Motorola, que conta com um acelerômetro integrado. A aplicação permite que o usuário realize um remapeamento dos gestos para qualquer função específica do aplicativo, o que reforça a natureza de personalização do algoritmo.

% Em sua conclusão 
\citeonline{uwave:2009} discorre sobre os desafios encontrados em definir um vocabulário de gestos que possam ser intuitivos para a grande maioria dos usuários. Outro ponto frisado é o fato do algoritmo depender apenas de um acelerômetro, o que dificulta no registro de certos gestos devido à falta de dados sobre a inclinação e o ângulo das forças realizada sobre o sensor, gerando uma certa confusão nos dados dos movimentos.

Diferentemente do uWave, a Hand.io utiliza irá utilizar um módulo MPU $6050$ \cite{invensense} que conta com um acelerômetro e um giroscópio integrados, o que permite que seja realizada uma medição mais precisa da direção dos movimentos realizados com a luva. \citeonline{uwave:2009} demonstra que existe uma variedade enorme de algoritmos que podem ser utilizados para reconhecimento de gestos. Diferentes algoritmos apresentam diferentes resultados variando com \textit{training sets} de tamanhos diferentes, logo na Hand.io serão realizados testes de desempenho com diversos classificadores até que se atinja um resultado ótimo.

\begin{comment}


\begin{table}[ht]
	\centering
    \caption{Comparação entre Hand.io e uWave}
    \label{tab:comp_uwave}
	\begin{tabular}{|l|c|c|c|}
    	\hline
        %\rowcolor{\color{gray}}
		\textbf{Recursos} & \textbf{Hand.io} & \textbf{uWave controle} \\
% 		& \textbf{uWave autenticação} \\
        \hline
        \hline
        Preso à mão & x & x \\
%         & \\
        \hline
        Utiliza botões &  & x \\
%         & \\
        \hline
        Captura contínua de sinais & x & \\
%         & \\
        \hline
        \hline
        \textbf{Sensores} & & \\
%         & \\
        \hline
        \hline
        Acelerômetro & x & x \\
%         & \\
        \hline
        Giroscópio & x & x \\
%         & \\
        \hline
        Compasso eletrônico & & x \\
%         & \\
        \hline
        \hline
        \textbf{Contexto aplicado} & Residencial & Controle \\
%         & Autenticação\\
        \hline
	\end{tabular}

\end{table}

\end{comment}

% \todo{fazer tabela de comparação}

%\todo[inline]{Descrever o que do trabalho apresentado pode ser aproveitado para o seu TCC}


%\section{A survey of glove-based systems and their applications}

%\cite{survey:2008}





